\documentclass{article}
\usepackage[margin=0.5in]{geometry}
\usepackage{amsmath}
\begin{document}

\section*{Workspace}

Start with a weighted directed graph $\mathbf{G}$ containing $n$ vertices $v \in \mathbf{V}$ with and $m$ edges $e \in \mathbf{E}$. \\

We start by defining a weight function for the vertices $F: \mathbf{V} \rightarrow \mathcal{R}$ with a corresponding weight value $f_i = F(v_i)$. 
Similarly we add a weight for each edge given by the weight function $W: \mathbf{E} \rightarrow \mathcal{R}$ with corresponding weights $w_i = W(e_i)$. 
Finally, define an activation function on each vertex taking an affine parameter $\lambda$ as $\mu\left[v\right]: \mathcal{R} x \mathcal{R} \rightarrow \mathcal{R}$. 
For simplicity let $\mu_i(\lambda) := \mu[v_i](\lambda)$.\\

Now we can define the potential energy of the system by:
\begin{equation}
\mathcal{E} = \sum_{i}\left[f_i - \mu_i(\sum_jw_jf_j, \lambda)\right]^2
\end{equation}

I am going to switch to Einstein notation, since I think it will really help make things easier. Values of weights will be represented with superscript.
We can then write the potential energy of the system using:

\begin{equation}
    \mathcal{E} = \sum_{i}\left[f_i - \mu_i(w^jf_j, \lambda)\right]^2
\end{equation}

One idea is to direct-sum the spaces of nodes and edges.
Each edge would also be a node, and the connection matrix becomes a matrix of 1's and 0's.
This allows writing the potential energy using:

\begin{equation}
    \mathcal{E} = \sum_{i}\left[f_i - \mu_i(\Delta^{jk}w_jf_k, \lambda)\right]^2
\end{equation}

Now we define a new vector $\mathbf{g}$ by combining $\mathbf{f}$ and $\mathbf{w}$:

\begin{eqnarray}
    g_i &=& f_i | i \in \{1...,n\} \\
    g_i &=& w_{i-n} | i \in \{n+1...,n+m\} \\
    \Delta^{ij} &\in & \{0, 1\}
\end{eqnarray}
\end{document}